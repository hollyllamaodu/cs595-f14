%%%%%%%%%%%%%%%%%%%%%%%%%%%%%%%%%%%%%%%%%
% Programming/Coding Assignment
% LaTeX Template
%
% This template has been downloaded from:
% http://www.latextemplates.com
%
% Original author:
% Ted Pavlic (http://www.tedpavlic.com)
%
% Note:
% The \lipsum[#] commands throughout this template generate dummy text
% to fill the template out. These commands should all be removed when 
% writing assignment content.
%
% This template uses a Perl script as an example snippet of code, most other
% languages are also usable. Configure them in the "CODE INCLUSION 
% CONFIGURATION" section.
%
%%%%%%%%%%%%%%%%%%%%%%%%%%%%%%%%%%%%%%%%%

%----------------------------------------------------------------------------------------
%	PACKAGES AND OTHER DOCUMENT CONFIGURATIONS
%----------------------------------------------------------------------------------------

\documentclass{article}
\usepackage{cite}

\usepackage{fancyhdr} % Required for custom headers
\usepackage{lastpage} % Required to determine the last page for the footer
\usepackage{extramarks} % Required for headers and footers
\usepackage[usenames,dvipsnames]{color} % Required for custom colors
\usepackage{graphicx} % Required to insert images
\usepackage{listings} % Required for insertion of code
\usepackage{courier} % Required for the courier font
\usepackage{lipsum} % Used for inserting dummy 'Lorem ipsum' text into the template
\usepackage{amsmath}
\usepackage{url}

% Margins
\topmargin=-0.45in
\evensidemargin=0in
\oddsidemargin=0in
\textwidth=6.5in
\textheight=9.0in
\headsep=0.25in

\linespread{1.1} % Line spacing

% Set up the header and footer
\pagestyle{fancy}
\lhead{\hmwkAuthorName} % Top left header
\chead{\hmwkClass\ (\hmwkClassInstructor\ \hmwkClassTime): \hmwkTitle} % Top center head
\rhead{\firstxmark} % Top right header
\lfoot{\lastxmark} % Bottom left footer
\cfoot{} % Bottom center footer
\rfoot{Page\ \thepage\ of\ \protect\pageref{LastPage}} % Bottom right footer
\renewcommand\headrulewidth{0.4pt} % Size of the header rule
\renewcommand\footrulewidth{0.4pt} % Size of the footer rule

\setlength\parindent{0pt} % Removes all indentation from paragraphs

%----------------------------------------------------------------------------------------
%	CODE INCLUSION CONFIGURATION
%----------------------------------------------------------------------------------------

\definecolor{MyDarkGreen}{rgb}{0.0,0.4,0.0} % This is the color used for comments
\lstloadlanguages{Perl} % Load Perl syntax for listings, for a list of other languages supported see: ftp://ftp.tex.ac.uk/tex-archive/macros/latex/contrib/listings/listings.pdf
\lstset{language=Perl, % Use Perl in this example
        frame=single, % Single frame around code
        basicstyle=\small\ttfamily, % Use small true type font
        keywordstyle=[1]\color{Blue}\bf, % Perl functions bold and blue
        keywordstyle=[2]\color{Purple}, % Perl function arguments purple
        keywordstyle=[3]\color{Blue}\underbar, % Custom functions underlined and blue
        identifierstyle=, % Nothing special about identifiers                                         
        commentstyle=\usefont{T1}{pcr}{m}{sl}\color{MyDarkGreen}\small, % Comments small dark green courier font
        stringstyle=\color{Purple}, % Strings are purple
        showstringspaces=false, % Don't put marks in string spaces
        tabsize=5, % 5 spaces per tab
        %
        % Put standard Perl functions not included in the default language here
        morekeywords={rand},
        %
        % Put Perl function parameters here
        morekeywords=[2]{on, off, interp},
        %
        % Put user defined functions here
        morekeywords=[3]{test},
       	%
        morecomment=[l][\color{Blue}]{...}, % Line continuation (...) like blue comment
        numbers=left, % Line numbers on left
        firstnumber=1, % Line numbers start with line 1
        numberstyle=\tiny\color{Blue}, % Line numbers are blue and small
        stepnumber=5 % Line numbers go in steps of 5
}

% Creates a new command to include a perl script, the first parameter is the filename of the script (without .pl), the second parameter is the caption
%\newcommand{\perlscript}[2]{
%\begin{itemize}
%\item[]\lstinputlisting[caption=#2,label=#1]{#1.pl}
%\end{itemize}
%}

%----------------------------------------------------------------------------------------
%	DOCUMENT STRUCTURE COMMANDS
%	Skip this unless you know what you're doing
%----------------------------------------------------------------------------------------

% Header and footer for when a page split occurs within a problem environment
\newcommand{\enterProblemHeader}[1]{
\nobreak\extramarks{#1}{#1 continued on next page\ldots}\nobreak
\nobreak\extramarks{#1 (continued)}{#1 continued on next page\ldots}\nobreak
}

% Header and footer for when a page split occurs between problem environments
\newcommand{\exitProblemHeader}[1]{
\nobreak\extramarks{#1 (continued)}{#1 continued on next page\ldots}\nobreak
\nobreak\extramarks{#1}{}\nobreak
}

\setcounter{secnumdepth}{0} % Removes default section numbers
\newcounter{homeworkProblemCounter} % Creates a counter to keep track of the number of problems

\newcommand{\homeworkProblemName}{}
\newenvironment{homeworkProblem}[1][Problem \arabic{homeworkProblemCounter}]{ % Makes a new environment called homeworkProblem which takes 1 argument (custom name) but the default is "Problem #"
\stepcounter{homeworkProblemCounter} % Increase counter for number of problems
\renewcommand{\homeworkProblemName}{#1} % Assign \homeworkProblemName the name of the problem
\section{\homeworkProblemName} % Make a section in the document with the custom problem count
\enterProblemHeader{\homeworkProblemName} % Header and footer within the environment
}{
\exitProblemHeader{\homeworkProblemName} % Header and footer after the environment
}

\newcommand{\problemAnswer}[1]{ % Defines the problem answer command with the content as the only argument
\noindent\framebox[\columnwidth][c]{\begin{minipage}{0.98\columnwidth}#1\end{minipage}} % Makes the box around the problem answer and puts the content inside
}

\newcommand{\homeworkSectionName}{}
\newenvironment{homeworkSection}[1]{ % New environment for sections within homework problems, takes 1 argument - the name of the section
\renewcommand{\homeworkSectionName}{#1} % Assign \homeworkSectionName to the name of the section from the environment argument
\subsection{\homeworkSectionName} % Make a subsection with the custom name of the subsection
\enterProblemHeader{\homeworkProblemName\ [\homeworkSectionName]} % Header and footer within the environment
}{
\enterProblemHeader{\homeworkProblemName} % Header and footer after the environment
}

%----------------------------------------------------------------------------------------
%	NAME AND CLASS SECTION
%----------------------------------------------------------------------------------------

\newcommand{\hmwkTitle}{Assignment\ \#10} % Assignment title
\newcommand{\hmwkDueDate}{Thursday,\ December 11,\ 2014} % Due date
\newcommand{\hmwkClass}{CS\ 595} % Course/class
\newcommand{\hmwkClassTime}{4:20pm} % Class/lecture time
\newcommand{\hmwkClassInstructor}{Dr. Nelson} % Teacher/lecturer
\newcommand{\hmwkAuthorName}{Holly Harkins} % Your name

%----------------------------------------------------------------------------------------
%	TITLE PAGE
%----------------------------------------------------------------------------------------

\title{
\vspace{2in}
\textmd{\textbf{\hmwkClass:\ \hmwkTitle}}\\
\normalsize\vspace{0.1in}\small{Due\ on\ \hmwkDueDate}\\
\vspace{0.1in}\large{\textit{\hmwkClassInstructor\ \hmwkClassTime}}
\vspace{3in}
}

\author{\textbf{\hmwkAuthorName}}
\date{} % Insert date here if you want it to appear below your name

%----------------------------------------------------------------------------------------

\begin{document}

\maketitle

%----------------------------------------------------------------------------------------
%	TABLE OF CONTENTS
%----------------------------------------------------------------------------------------

%\setcounter{tocdepth}{1} % Uncomment this line if you don't want subsections listed in the ToC

\newpage
\tableofcontents
\newpage

%----------------------------------------------------------------------------------------
%	PROBLEM 1
%----------------------------------------------------------------------------------------

% To have just one problem per page, simply put a \clearpage after each problem

\begin{homeworkProblem}
%Listing \ref{part1} shows a Perl script.
%\lipsum[1]

\begin{verbatim}

Choose a blog or a news feed (or something similar as long as it has
an Atom or RSS feed).  It should be on a topic or topics of which you
are qualified to provide classification training data.  In other words,
choose something that you enjoy and are knowledgeable of.  Find a feed
with at least 100 entries.

Create between four and eight different categories for the entries
in the feed:

examples: 
work, class, family, news, deals
liberal, conservative, moderate, libertarian
sports, local, financial, national, international, entertainment
metal, electronic, ambient, folk, hip-hop, pop

Download and process the pages of the feed as per the week 12 
class slides.

\end{verbatim}

\begin{verbatim}

Answer:
Python on my laptop was not working so I had to complete this assignment 
without my results.

I chose blog http://taylor-swift-love.blogspot.com/.  Using a curl 
statement I would grab the first 100 entries.  I used the categories 
below for the first 100 entries in the feed and classified them 
manually.

international
fashion 
awards
shows

http://www.taylor-swift-love.blogspot.com/feeds/posts/default?max-results=100

In Assignment10.py, the entries are looped through which generates the word counts for
each entry. Train the classifier with my chosen categories. Identifies a category for
each of them and outputs results.


\end{verbatim}

\lstinputlisting[breaklines=true, caption=Assigment 10 Python]{Assignment10.py}
\clearpage

\end{homeworkProblem}
\clearpage

%----------------------------------------------------------------------------------------
%	PROBLEM 2
%----------------------------------------------------------------------------------------

\begin{homeworkProblem}
%Listing \ref{part2} shows a Perl script.
%\lipsum[2]

\begin{verbatim}

Manually classify the first 50 entries, and then classify (using
the fisher classifier) the remaining 50 entries. Report the cprob()
values for the 50 titles as well.  From the title or entry itself,
specify the 1-, 2-, or 3-gram that you used for the string to
classify.  Do not repeat strings; you will have 50 unique strings.
For example, in these titles the string used is marked with *s:

*Rachel Goswell* - "Waves Are Universal" (LP Review) 
The *Naked and Famous* - "Passive Me, Aggressive You" (LP Review)
*Negativland* - "Live at Lewis's, Norfolk VA, November 21, 1992" (concert)
Negativland - "*U2*" (LP Review)

Note how "Negativland" is not repeated as a classification string.

Create a table with the title, the string used for classification,
cprob(), predicted category, and actual category.

\end{verbatim}

\begin{verbatim}

Answer:
From Assignment10.py, the function fisherclassifier classify the remaining 50
entries. It will get the predicted category for each entry, from fisher 
classifier, and train the classifier with each category predication. 

fisherclassifier function- The frequency of this feature in all the categories. 
The probability is the frequency in this category divided by the overall 
frequency. Loop through looking for the best result and make sure it exceeds its 
minimum.

fisherprob funtion-Multiply all the probabilities together.  Take the natural 
log and multiply by -2. Use the Inverse chi-squared function to get a probability.

Table would be displayed here showing Classifier Data: title, classifier, 
predicted, actual, and the cprob() .

\end{verbatim}


\end{homeworkProblem}
\newpage


\clearpage

%----------------------------------------------------------------------------------------
%	PROBLEM 3
%----------------------------------------------------------------------------------------

\begin{homeworkProblem}
%Listing \ref{part3} shows a Perl script.
%\lipsum[3]

\begin{verbatim}

Assess the performance of your classifier in each of your categories
by computing precision, recall, and F1.  Note that the definitions
of precisions and recall are slightly different in the context of
classification; see:

http://en.wikipedia.org/wiki/Precision_and_recall#Definition_.28classification_context.29

and

http://en.wikipedia.org/wiki/F1_score
\end{verbatim}

\begin{verbatim}

Answer: The results of precision and recall would be displayed here.
Depending on if the entries were categorized correctly would determine if the
prediction accuracy is the accuracy based off its prediction, or that 
category versus the actual.  False positives could be determined if incorrect
classifiers were used.

\end{verbatim}

\begin{table}[ht]

\centering

\begin{tabular}{ c | c | c }
\hline
     &  precision & recall\\ \hline\hline
    
international   & 0 & 0 \\
fashion & 0 & 0 \\
awards & 0 & 0 \\
shows & 0 & 0 \\
\hline
\end{tabular}

\caption{Preformance Measures}

\end{table}

\end{homeworkProblem}
\newpage

\clearpage

%----------------------------------------------------------------------------------------
%	PROBLEM 4
%----------------------------------------------------------------------------------------

\begin{homeworkProblem}
%Listing \ref{part4} shows a Perl script.
%\lipsum[4]

\begin{verbatim}

Redo questions 2 & 3, but with manually train 90 entries and 
then classify the remaining 10.

Then redo questions 2 & 3, but with the extensions on slide 26
and pp. 136--138.  Fully discuss the changes you've made.

Which method (more training vs. better features) gave better improvement
over your baseline?  Why do you think that is?

\end{verbatim}


\end{homeworkProblem}
\newpage


\clearpage

%-------------------------------------------------------------------
%References
%-------------------------------------------------------------------
\bibliographystyle{plain}
\bibliography{assignment7}


%----------------------------------------------------------------------------------------

\end{document}